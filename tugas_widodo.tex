
%% bare_jrnl.tex
%% V1.3
%% 2007/01/11
%% by Michael Shell
%% see http://www.michaelshell.org/
%% for current contact information.
%%
%% This is a skeleton file demonstrating the use of IEEEtran.cls
%% (requires IEEEtran.cls version 1.7 or later) with an IEEE journal paper.
%%
%% Support sites:
%% http://www.michaelshell.org/tex/ieeetran/
%% http://www.ctan.org/tex-archive/macros/latex/contrib/IEEEtran/
%% and
%% http://www.ieee.org/



% *** Authors should verify (and, if needed, correct) their LaTeX system  ***
% *** with the testflow diagnostic prior to trusting their LaTeX platform ***
% *** with production work. IEEE's font choices can trigger bugs that do  ***
% *** not appear when using other class files.                            ***
% The testflow support page is at:
% http://www.michaelshell.org/tex/testflow/


%%*************************************************************************
%% Legal Notice:
%% This code is offered as-is without any warranty either expressed or
%% implied; without even the implied warranty of MERCHANTABILITY or
%% FITNESS FOR A PARTICULAR PURPOSE! 
%% User assumes all risk.
%% In no event shall IEEE or any contributor to this code be liable for
%% any damages or losses, including, but not limited to, incidental,
%% consequential, or any other damages, resulting from the use or misuse
%% of any information contained here.
%%
%% All comments are the opinions of their respective authors and are not
%% necessarily endorsed by the IEEE.
%%
%% This work is distributed under the LaTeX Project Public License (LPPL)
%% ( http://www.latex-project.org/ ) version 1.3, and may be freely used,
%% distributed and modified. A copy of the LPPL, version 1.3, is included
%% in the base LaTeX documentation of all distributions of LaTeX released
%% 2003/12/01 or later.
%% Retain all contribution notices and credits.
%% ** Modified files should be clearly indicated as such, including  **
%% ** renaming them and changing author support contact information. **
%%
%% File list of work: IEEEtran.cls, IEEEtran_HOWTO.pdf, bare_adv.tex,
%%                    bare_conf.tex, bare_jrnl.tex, bare_jrnl_compsoc.tex
%%*************************************************************************

% Note that the a4paper option is mainly intended so that authors in
% countries using A4 can easily print to A4 and see how their papers will
% look in print - the typesetting of the document will not typically be
% affected with changes in paper size (but the bottom and side margins will).
% Use the testflow package mentioned above to verify correct handling of
% both paper sizes by the user's LaTeX system.
%
% Also note that the "draftcls" or "draftclsnofoot", not "draft", option
% should be used if it is desired that the figures are to be displayed in
% draft mode.
%
\documentclass[conference]{IEEEtran}
%\documentclass[a4paper]{article}
% If IEEEtran.cls has not been installed into the LaTeX system files,
% manually specify the path to it like:
% \documentclass[journal]{../sty/IEEEtran}





% Some very useful LaTeX packages include:
% (uncomment the ones you want to load)

\usepackage{multirow}
\usepackage{hhline}
% *** MISC UTILITY PACKAGES ***
%
%\usepackage{ifpdf}
% Heiko Oberdiek's ifpdf.sty is very useful if you need conditional
% compilation based on whether the output is pdf or dvi.
% usage:
% \ifpdf
%   % pdf code
% \else
%   % dvi code
% \fi
% The latest version of ifpdf.sty can be obtained from:
% http://www.ctan.org/tex-archive/macros/latex/contrib/oberdiek/
% Also, note that IEEEtran.cls V1.7 and later provides a builtin
% \ifCLASSINFOpdf conditional that works the same way.
% When switching from latex to pdflatex and vice-versa, the compiler may
% have to be run twice to clear warning/error messages.



% *** CITATION PACKAGES ***
%
\usepackage{cite}
% cite.sty was written by Donald Arseneau
% V1.6 and later of IEEEtran pre-defines the format of the cite.sty package
% \cite{} output to follow that of IEEE. Loading the cite package will
% result in citation numbers being automatically sorted and properly
% "compressed/ranged". e.g., [1], [9], [2], [7], [5], [6] without using
% cite.sty will become [1], [2], [5]--[7], [9] using cite.sty. cite.sty's
% \cite will automatically add leading space, if needed. Use cite.sty's
% noadjust option (cite.sty V3.8 and later) if you want to turn this off.
% cite.sty is already installed on most LaTeX systems. Be sure and use
% version 4.0 (2003-05-27) and later if using hyperref.sty. cite.sty does
% not currently provide for hyperlinked citations.
% The latest version can be obtained at:
% http://www.ctan.org/tex-archive/macros/latex/contrib/cite/
% The documentation is contained in the cite.sty file itself.


\usepackage{graphicx}
%\usepackage[parfill]{parskip}  
\setlength{\parskip}{0.8em}
%\usepackage[]{mcode}

%\usepackage{apacite}
% *** GRAPHICS RELATED PACKAGES ***
%
\ifCLASSINFOpdf
  % \usepackage[pdftex]{graphicx}
  % declare the path(s) where your graphic files are
  % \graphicspath{{../pdf/}{../jpeg/}}
  % and their extensions so you won't have to specify these with
  % every instance of \includegraphics
  % \DeclareGraphicsExtensions{.pdf,.jpeg,.png}
\else
  % or other class option (dvipsone, dvipdf, if not using dvips). graphicx
  % will default to the driver specified in the system graphics.cfg if no
  % driver is specified.
  % \usepackage[dvips]{graphicx}
  % declare the path(s) where your graphic files are
  % \graphicspath{{../eps/}}
  % and their extensions so you won't have to specify these with
  % every instance of \includegraphics
  % \DeclareGraphicsExtensions{.eps}
\fi
% graphicx was written by David Carlisle and Sebastian Rahtz. It is
% required if you want graphics, photos, etc. graphicx.sty is already
% installed on most LaTeX systems. The latest version and documentation can
% be obtained at: 
% http://www.ctan.org/tex-archive/macros/latex/required/graphics/
% Another good source of documentation is "Using Imported Graphics in
% LaTeX2e" by Keith Reckdahl which can be found as epslatex.ps or
% epslatex.pdf at: http://www.ctan.org/tex-archive/info/
%
% latex, and pdflatex in dvi mode, support graphics in encapsulated
% postscript (.eps) format. pdflatex in pdf mode supports graphics
% in .pdf, .jpeg, .png and .mps (metapost) formats. Users should ensure
% that all non-photo figures use a vector format (.eps, .pdf, .mps) and
% not a bitmapped formats (.jpeg, .png). IEEE frowns on bitmapped formats
% which can result in "jaggedy"/blurry rendering of lines and letters as
% well as large increases in file sizes.
%
% You can find documentation about the pdfTeX application at:
% http://www.tug.org/applications/pdftex





% *** MATH PACKAGES ***
%
\usepackage{amsmath}
% A popular package from the American Mathematical Society that provides
% many useful and powerful commands for dealing with mathematics. If using
% it, be sure to load this package with the cmex10 option to ensure that
% only type 1 fonts will utilized at all point sizes. Without this option,
% it is possible that some math symbols, particularly those within
% footnotes, will be rendered in bitmap form which will result in a
% document that can not be IEEE Xplore compliant!
%
% Also, note that the amsmath package sets \interdisplaylinepenalty to 10000
% thus preventing page breaks from occurring within multiline equations. Use:
%\interdisplaylinepenalty=2500
% after loading amsmath to restore such page breaks as IEEEtran.cls normally
% does. amsmath.sty is already installed on most LaTeX systems. The latest
% version and documentation can be obtained at:
% http://www.ctan.org/tex-archive/macros/latex/required/amslatex/math/





% *** SPECIALIZED LIST PACKAGES ***
%
%\usepackage{algorithmic}
% algorithmic.sty was written by Peter Williams and Rogerio Brito.
% This package provides an algorithmic environment fo describing algorithms.
% You can use the algorithmic environment in-text or within a figure
% environment to provide for a floating algorithm. Do NOT use the algorithm
% floating environment provided by algorithm.sty (by the same authors) or
% algorithm2e.sty (by Christophe Fiorio) as IEEE does not use dedicated
% algorithm float types and packages that provide these will not provide
% correct IEEE style captions. The latest version and documentation of
% algorithmic.sty can be obtained at:
% http://www.ctan.org/tex-archive/macros/latex/contrib/algorithms/
% There is also a support site at:
% http://algorithms.berlios.de/index.html
% Also of interest may be the (relatively newer and more customizable)
% algorithmicx.sty package by Szasz Janos:
% http://www.ctan.org/tex-archive/macros/latex/contrib/algorithmicx/




% *** ALIGNMENT PACKAGES ***
%
%\usepackage{array}
% Frank Mittelbach's and David Carlisle's array.sty patches and improves
% the standard LaTeX2e array and tabular environments to provide better
% appearance and additional user controls. As the default LaTeX2e table
% generation code is lacking to the point of almost being broken with
% respect to the quality of the end results, all users are strongly
% advised to use an enhanced (at the very least that provided by array.sty)
% set of table tools. array.sty is already installed on most systems. The
% latest version and documentation can be obtained at:
% http://www.ctan.org/tex-archive/macros/latex/required/tools/


%\usepackage{mdwmath}
%\usepackage{mdwtab}
% Also highly recommended is Mark Wooding's extremely powerful MDW tools,
% especially mdwmath.sty and mdwtab.sty which are used to format equations
% and tables, respectively. The MDWtools set is already installed on most
% LaTeX systems. The lastest version and documentation is available at:
% http://www.ctan.org/tex-archive/macros/latex/contrib/mdwtools/


% IEEEtran contains the IEEEeqnarray family of commands that can be used to
% generate multiline equations as well as matrices, tables, etc., of high
% quality.


%\usepackage{eqparbox}
% Also of notable interest is Scott Pakin's eqparbox package for creating
% (automatically sized) equal width boxes - aka "natural width parboxes".
% Available at:
% http://www.ctan.org/tex-archive/macros/latex/contrib/eqparbox/





% *** SUBFIGURE PACKAGES ***
%\usepackage[tight,footnotesize]{subfigure}
% subfigure.sty was written by Steven Douglas Cochran. This package makes it
% easy to put subfigures in your figures. e.g., "Figure 1a and 1b". For IEEE
% work, it is a good idea to load it with the tight package option to reduce
% the amount of white space around the subfigures. subfigure.sty is already
% installed on most LaTeX systems. The latest version and documentation can
% be obtained at:
% http://www.ctan.org/tex-archive/obsolete/macros/latex/contrib/subfigure/
% subfigure.sty has been superceeded by subfig.sty.



%\usepackage[caption=false]{caption}
%\usepackage[font=footnotesize]{subfig}
% subfig.sty, also written by Steven Douglas Cochran, is the modern
% replacement for subfigure.sty. However, subfig.sty requires and
% automatically loads Axel Sommerfeldt's caption.sty which will override
% IEEEtran.cls handling of captions and this will result in nonIEEE style
% figure/table captions. To prevent this problem, be sure and preload
% caption.sty with its "caption=false" package option. This is will preserve
% IEEEtran.cls handing of captions. Version 1.3 (2005/06/28) and later 
% (recommended due to many improvements over 1.2) of subfig.sty supports
% the caption=false option directly:
%\usepackage[caption=false,font=footnotesize]{subfig}
%
% The latest version and documentation can be obtained at:
% http://www.ctan.org/tex-archive/macros/latex/contrib/subfig/
% The latest version and documentation of caption.sty can be obtained at:
% http://www.ctan.org/tex-archive/macros/latex/contrib/caption/




% *** FLOAT PACKAGES ***
%
%\usepackage{fixltx2e}
% fixltx2e, the successor to the earlier fix2col.sty, was written by
% Frank Mittelbach and David Carlisle. This package corrects a few problems
% in the LaTeX2e kernel, the most notable of which is that in current
% LaTeX2e releases, the ordering of single and double column floats is not
% guaranteed to be preserved. Thus, an unpatched LaTeX2e can allow a
% single column figure to be placed prior to an earlier double column
% figure. The latest version and documentation can be found at:
% http://www.ctan.org/tex-archive/macros/latex/base/



%\usepackage{stfloats}
% stfloats.sty was written by Sigitas Tolusis. This package gives LaTeX2e
% the ability to do double column floats at the bottom of the page as well
% as the top. (e.g., "\begin{figure*}[!b]" is not normally possible in
% LaTeX2e). It also provides a command:
%\fnbelowfloat
% to enable the placement of footnotes below bottom floats (the standard
% LaTeX2e kernel puts them above bottom floats). This is an invasive package
% which rewrites many portions of the LaTeX2e float routines. It may not work
% with other packages that modify the LaTeX2e float routines. The latest
% version and documentation can be obtained at:
% http://www.ctan.org/tex-archive/macros/latex/contrib/sttools/
% Documentation is contained in the stfloats.sty comments as well as in the
% presfull.pdf file. Do not use the stfloats baselinefloat ability as IEEE
% does not allow \baselineskip to stretch. Authors submitting work to the
% IEEE should note that IEEE rarely uses double column equations and
% that authors should try to avoid such use. Do not be tempted to use the
% cuted.sty or midfloat.sty packages (also by Sigitas Tolusis) as IEEE does
% not format its papers in such ways.


%\ifCLASSOPTIONcaptionsoff
%  \usepackage[nomarkers]{endfloat}
% \let\MYoriglatexcaption\caption
% \renewcommand{\caption}[2][\relax]{\MYoriglatexcaption[#2]{#2}}
%\fi
% endfloat.sty was written by James Darrell McCauley and Jeff Goldberg.
% This package may be useful when used in conjunction with IEEEtran.cls'
% captionsoff option. Some IEEE journals/societies require that submissions
% have lists of figures/tables at the end of the paper and that
% figures/tables without any captions are placed on a page by themselves at
% the end of the document. If needed, the draftcls IEEEtran class option or
% \CLASSINPUTbaselinestretch interface can be used to increase the line
% spacing as well. Be sure and use the nomarkers option of endfloat to
% prevent endfloat from "marking" where the figures would have been placed
% in the text. The two hack lines of code above are a slight modification of
% that suggested by in the endfloat docs (section 8.3.1) to ensure that
% the full captions always appear in the list of figures/tables - even if
% the user used the short optional argument of \caption[]{}.
% IEEE papers do not typically make use of \caption[]'s optional argument,
% so this should not be an issue. A similar trick can be used to disable
% captions of packages such as subfig.sty that lack options to turn off
% the subcaptions:
% For subfig.sty:
% \let\MYorigsubfloat\subfloat
% \renewcommand{\subfloat}[2][\relax]{\MYorigsubfloat[]{#2}}
% For subfigure.sty:
% \let\MYorigsubfigure\subfigure
% \renewcommand{\subfigure}[2][\relax]{\MYorigsubfigure[]{#2}}
% However, the above trick will not work if both optional arguments of
% the \subfloat/subfig command are used. Furthermore, there needs to be a
% description of each subfigure *somewhere* and endfloat does not add
% subfigure captions to its list of figures. Thus, the best approach is to
% avoid the use of subfigure captions (many IEEE journals avoid them anyway)
% and instead reference/explain all the subfigures within the main caption.
% The latest version of endfloat.sty and its documentation can obtained at:
% http://www.ctan.org/tex-archive/macros/latex/contrib/endfloat/
%
% The IEEEtran \ifCLASSOPTIONcaptionsoff conditional can also be used
% later in the document, say, to conditionally put the References on a 
% page by themselves.





% *** PDF, URL AND HYPERLINK PACKAGES ***
%
%\usepackage{url}
% url.sty was written by Donald Arseneau. It provides better support for
% handling and breaking URLs. url.sty is already installed on most LaTeX
% systems. The latest version can be obtained at:
% http://www.ctan.org/tex-archive/macros/latex/contrib/misc/
% Read the url.sty source comments for usage information. Basically,
% \url{my_url_here}.





% *** Do not adjust lengths that control margins, column widths, etc. ***
% *** Do not use packages that alter fonts (such as pslatex).         ***
% There should be no need to do such things with IEEEtran.cls V1.6 and later.
% (Unless specifically asked to do so by the journal or conference you plan
% to submit to, of course. )


% correct bad hyphenation here
\hyphenation{op-tical net-works semi-conduc-tor}


\begin{document}
%
% paper title
% can use linebreaks \\ within to get better formatting as desired
%
%
% author names and IEEE memberships
% note positions of commas and nonbreaking spaces ( ~ ) LaTeX will not break
% a structure at a ~ so this keeps an author's name from being broken across
% two lines.
% use \thanks{} to gain access to the first footnote area
% a separate \thanks must be used for each paragraph as LaTeX2e's \thanks
% was not built to handle multiple paragraphs
%
\title{Cascading of K-Means and K-NN Methods \\ for Texture Image Classification}
%\author{B. Ari Kuncoro\\Master of Information Technology\\Bina Nusantara University\\
%Jakarta, Indonesia \\ b.kuncoro@binus.ac.id}

% Activate to display a given date or no date
\author{
\IEEEauthorblockN{B. Ari Kuncoro}
\IEEEauthorblockA{Master of Information Technology\\
Bina Nusantara University\\
Jakarta, Indonesia\\
Email: b.kuncoro [at] binus.ac.id}
\and
\IEEEauthorblockN{Filo Limantara}
\IEEEauthorblockA{Master of Information Technology\\
Bina Nusantara University\\
Jakarta, Indonesia\\
Email: filo.limantara [at] gmail.com}\\[0.9cm]  %<------- Extra vertical space
}


% note the % following the last \IEEEmembership and also \thanks - 
% these prevent an unwanted space from occurring between the last author name
% and the end of the author line. i.e., if you had this:
% 
% \author{....lastname \thanks{...} \thanks{...} }
%                     ^------------^------------^----Do not want these spaces!
%
% a space would be appended to the last name and could cause every name on that
% line to be shifted left slightly. This is one of those "LaTeX things". For
% instance, "\textbf{A} \textbf{B}" will typeset as "A B" not "AB". To get
% "AB" then you have to do: "\textbf{A}\textbf{B}"
% \thanks is no different in this regard, so shield the last } of each \thanks
% that ends a line with a % and do not let a space in before the next \thanks.
% Spaces after \IEEEmembership other than the last one are OK (and needed) as
% you are supposed to have spaces between the names. For what it is worth,
% this is a minor point as most people would not even notice if the said evil
% space somehow managed to creep in.

% The paper headers
%\markboth{SCI 1 Assignment 3 - ANFIS}%
%{Shell \MakeLowercase{\textit{et al.}}: Bare Demo of IEEEtran.cls for Journals}
% The only time the second header will appear is for the odd numbered pages
% after the title page when using the twoside option.
% 
% *** Note that you probably will NOT want to include the author's ***
% *** name in the headers of peer review papers.                   ***
% You can use \ifCLASSOPTIONpeerreview for conditional compilation here if
% you desire.

% If you want to put a publisher's ID mark on the page you can do it like
% this:
%\IEEEpubid{0000--0000/00\$00.00~\copyright~2007 IEEE}
% Remember, if you use this you must call \IEEEpubidadjcol in the second
% column for its text to clear the IEEEpubid mark.

% use for special paper notices
%\IEEEspecialpapernotice{(Invited Paper)}

% make the title area
\maketitle


\begin{abstract}
%\boldmath
One of the most important problems in pattern recognition is texture-based image classification. It is useful in recognition and interpretation, because as image, it consists of related and interrelated elements. In this work, to extract feature, Gray Level Co-Occurrence Matrix (GLCM) methods was applied. The cascading of K-Means clustering and K-Nearast Neighbor (K-NN) classification is proposed to classify images into three-class texture image. The objective of this work is to check whether the cascading methods of K-Means and K-NN can increase the accuracy compare with the original K-NN. The dataset has original labels and they were used to get the accuracy parameter for comparison between K-NN and 'cascade K-Means and KNN'. 

\end{abstract}
% IEEEtran.cls defaults to using nonbold math in the Abstract.
% This preserves the distinction between vectors and scalars. However,
% if the journal you are submitting to favors bold math in the abstract,
% then you can use LaTeX's standard command \boldmath at the very start
% of the abstract to achieve this. Many IEEE journals frown on math
% in the abstract anyway.

\renewcommand\IEEEkeywordsname{Keywords}
% Note that keywords are not normally used for peerreview papers.
\begin{IEEEkeywords}
\textit{Texture Images; K-Nearest Neighbor; K-Means; Cascade}
\end{IEEEkeywords}

% For peer review papers, you can put extra information on the cover
% page as needed:
% \ifCLASSOPTIONpeerreview
% \begin{center} \bfseries EDICS Category: 3-BBND \end{center}
% \fi
%
% For peerreview papers, this IEEEtran command inserts a page break and
% creates the second title. It will be ignored for other modes.
\IEEEpeerreviewmaketitle
\section{Introduction}
% The very first letter is a 2 line initial drop letter followed
% by the rest of the first word in caps.
% 
% form to use if the first word consists of a single letter:
% \IEEEPARstart{A}{demo} file is ....
% 
% form to use if you need the single drop letter followed by
% normal text (unknown if ever used by IEEE):
% \IEEEPARstart{A}{}demo file is ....
% 
% Some journals put the first two words in caps:
% \IEEEPARstart{T}{his demo} file is ....
% 
% Here we have the typical use of a "T" for an initial drop letter
% and "HIS" in caps to complete the first word.
\IEEEPARstart {O}{ne} of the well-known low-level image features is texture \cite{Avci2015}. It is useful in recognition and interpretation, because as image, it consists of related and interrelated elements \cite{Chandankhede2012}. Several steps must be done to extract features from the texture images. Many researchers has reported various methods in how to extract them. They include statistical methods of Gray Level Co-occurrence Matrix (GLCM), energy filters and edgeness factor, Gabor filters, and Wavelet transform \cite{Ruiz2004,Mohanaiah2013}. There is no strict guideline to determine what features are most suitable for specific data, thus experimental methods such as empirical and heuristic methods are commonly used.\par

Common classification methods were proposed such as K-NN, Adaptive Neuro Fuzzy Inference System (ANFIS), Support Vector Machne (SVM), etc. Different from those pure-classifier, Karegowda, Jayaram, and Manjunath's \cite{Karegowda2012} performed classification towards diabetic cascade the K-Means clustering and K-NN. The result shows that cascading K-Means clustering and K-NN classification give best accuracy compare with other report results. In this work, we perform similar technique to Karegowda et al., in which they used cascading of K-Means clustering and K-Nearest Neighbor. 

\section{Dataset}

\graphicspath{ {images/} }

\begin{figure}[!t]
\centering
\includegraphics[width=2.5 in]{texture_tugas.jpg}
\caption{Samples of Dataset}
\label{fig:texture_fig}
\end{figure}
The image dataset can be downloaded at http://www.nada.kth.se/cvap/databases/kth-tips/download.html \cite{KTH}. Three-class texture image samples consist of alumunium foil, corduroy, and orange peel are illustrated in Fig. \ref{fig:texture_fig}. Each class contains 80 texture images, thus there were 240 texture images collected. 

\section{Methodology}
The methodology of this study is illustrated in Fig. \ref{fig:flowchart}. First, 240 gray-scale images were collected in one folder. Then the images were resized into $64\times 64$ pixels. On each images, GLCM block was applied thus GLCM matrix was produced. From GLCM, energy and entropy were extracted as attribute 1 and attribute 2. The original labels of the data were removed first, thus the data with ($240\times2$) size became the input of unsupervised learning K-Means clustering (in which the K equals to 3). The categorization result of K-Means clustering is considered as labels for K-NN classifier training purpose. Data was split into 50\% for training and 50\% for testing using interleaving. The result of testing towards the trained model is then calculated using confusion matrix, in which the predicted data is compared with the true data of original label. 

\begin{figure}[!t]
\centering
\includegraphics[width=1.3 in]{flowchart-tugas.jpg}
\caption{Flowchart of Classification System}
\label{fig:flowchart}
\end{figure}

\subsection{Gray Level Co-Occurrence Matrix (GCLM)}

Gray Level Co-Occurrence Matrix (GLCM) has proved to be a popular statistical method of extracting textural feature from images. According to co-occurrence matrix, Haralick defines fourteen textural features measured from the probability matrix to extract the characteristics of texture statistics of remote sensing images \cite{Costianes}. Mohanaiah \cite{Mohanaiah2013} performed extraction using these important features, Angular Second Moment (energy), Correlation, Entropy, and the Inverse Difference Moment. In this work, feature of Entropy and Energy from GLCM Matrix were used as attribute 3 and attribute 4 respectively. 

\subsubsection{Angular Second Moment (Energy)}
Angular Second Moment or Energy is the sum of squares of entries in the GLCM matrix. It is also known as homogeneity. The value of which energy high when image has very good homogeneity or when pixels are very similar \cite{Mohanaiah2013}. 

\begin{equation}
Energy = \sum_{i=0}^{Ng-1}\sum_{j=0}^{Ng-1}P(i,j)^2
\end{equation}

\subsubsection{Entropy}
Entropy shows the amount of information of the image that is needed for the image compression. Entropy measures the loss of information or message in a transmitted signal and also measures the image information. 

\begin{equation}
Entropy = \sum_{i=0}^{Ng-1}\sum_{j=0}^{Ng-1}-P(i,j)\times log(P(i,j))
\end{equation}

In this paper, each class was represented by integer. For combination 1, integer 1, 2, and 3 represent aluminium foil, corduroy, and orange peel respectively. Since the feature extraction process produced four features, each of the data contains four attributes. As a result, the used data contains four inputs and one output. The snippet of the data is as follows. 

\begin{figure}[!t]
\centering
\includegraphics[width=2.3 in]{data_in.jpg}
\caption{Data of attribute 1, attribute 2, and Class}
\label{fig:data_feature_class}
\end{figure}



\subsection{K-Means Clustering}
K-means algorithm is a clustering algorithm that is invented by LLoyd \cite{Lloyd1982}.The aim of the K-means algorithm is to divide M points in N dimensions into K clusters so that the within-cluster sum of squares is minimize \cite{Hartigan1979}. K-means algorithm uses an iterative technique to minimize the sum of squares.

Given an initial set of k means m1$^{\left(1\right)}$
,…,mk$^{\left(1\right)}$ (see below), the algorithm consist of two phase:\newline

$\boldsymbol{Assignment\ step}$: Assign each observation to the cluster whose mean yields the least within-cluster sum of squares (WCSS). Since the sum of squares is the squared Euclidean distance, this is intuitively the "nearest" mean.
\begin{equation}
S_i^{(t)} = \big \{ x_p : \big \| x_p - m^{(t)}_i \big \|^2 \le \big \| x_p - m^{(t)}_j \big \|^2 \ \forall j, 1 \le j \le k \big\}
\end{equation}
where each $x_p $is assigned to exactly one $S^{(t)}$, even if it could be assigned to two or more of them.\newline
$\boldsymbol{Update\ step}$: Calculate the new means to be the centroids of the observations in the new clusters.
\begin{equation}
m^{(t+1)}_i = \frac{1}{|S^{(t)}_i|} \sum_{x_j \in S^{(t)}_i} x_j
\end{equation}

\subsection{K-Nearest Neighbor}
K-nearest-neighbor (kNN) classification is one of the most fundamental and simple classification algorithm. It should the first choices for a classification study if we have no prior knowledge about the data. It was first proposed by Fix and Hodges \cite{EvelynFix1951}. 

The k-nearest-neighbor classifier is commonly based on the Euclidean distance between a test sample and the specified training samples. Let $x_i$ be an input sample with p features ($x_{i1}$,$x_{i2}$,…,$x_{ip}$) , n be the total number of input samples (i=1,2,…,n) and p the total number of features (j=1,2,…,p) . The Euclidean distance between sample $x_i$ and $x_l$ (l=1,2,…,n) is defined as

\begin{equation}
d({\textbf x}_i,{\textbf x}_l) = \sqrt{(x_{i1}-x_{l1})^2 + (x_{i2}-x_{l2})^2 + \cdots + (x_{ip}-x_{lp})^2}
\end{equation}

After calculating the distance of training samples to input sample $x_i$, select \textit{k} closest sample $x_m$, and we finally can base the assignment of a label upon $x_i$ by the maximum predominance of a particular class in its neighborhood \cite{Cover1967}.

\section{Results and Discussion}

To visualize the data based on generated attributes, 2-D plotting data graph is required. It is depicted in Fig. \ref{fig:plotinputdata} that color of blue, green, and red represent aluminium foil, corduroy, and orange peel textures respectively. It is seen that aluminium foil is well clustered on the right bottom of the cartesian graph. While corduroy and orange peel seems hard to be separated. 

\begin{figure}[!t]
\centering
\includegraphics[width=2.8 in]{plot_data_tugas.jpg}
\caption{Plot Input data in 2-D}
\label{fig:plotinputdata}
\end{figure}

\begin{figure}[!t]
\centering
\includegraphics[width=3.2 in]{confmat_knn.jpg}
\caption{Confusion matrix result of K-NN}
\label{fig:confmatrix-knn}
\end{figure}

\begin{figure}[!t]
\centering
\includegraphics[width=2.1 in]{confmat_k-means.jpg}
\caption{Confusion matrix result of K-Means}
\label{fig:confmatrix-k-means}
\end{figure}

\begin{figure}[!t]
\centering
\includegraphics[width=3.2 in]{confmat_cascade.jpg}
\caption{Confusion matrix result of Cascade of K-Means and K-NN}
\label{fig:confmatrix-cascade}
\end{figure}



Precision-recall result and accuracy of each methods are shown in Fig. \ref{fig:confmatrix-knn}, \ref{fig:confmatrix-k-means}, and \ref{fig:confmatrix-cascade}. 

The accuracy, precision, recall, and f1-score are calculated according this the following formula \cite{Olson2008}:\newline

\begin{equation}
\text{accuracy} = \frac{TP+TN}{TP+FP+FN+TN} 
\end{equation}

\begin{equation}
\text{recall} = \frac{TP}{TP+FN} 
\end{equation}

\begin{equation}
\text{precision}=\frac{TP}{TP+FP}
\end{equation}

\begin{equation}
F_1 = 2 \cdot \frac{\mathrm{precision} \cdot \mathrm{recall}}{\mathrm{precision} + \mathrm{recall}}.
\end{equation}

where TP are number of true positive, TN are number of true negative, FP are number of false positive, and FN are number of false negative.

% \begin{equation}
% Accuracy = \frac{TP+TN}{TP + FN + TN + FP}\times100\%
% \label{eq:accuracy}
% \end{equation}

% Where, 

% \begin{itemize}
%   \item True Positive (TP) represents the number of positive data that is correctly labeled.
%   \item True Negative (TN) represents the number of negative data that is correctly labeled.
%   \item False Positive (FP) represents the number of positive data that is incorrectly labeled.
%   \item False Negative (FN) represents the number of negative data that is incorrectly labeled.
% \end{itemize}

The performance result of K-NN and cascade of K-Means and K-NN can be seen in Table \ref{table_result}. It seen that the accuracy K-NN is 87\%, while the accuracy of K-Means and K-NN is 88\% or there is 1\% increase of accuracy. 

\begin{table}[!t]
% increase table row spacing, adjust to taste
\renewcommand{\arraystretch}{1.3}
% if using array.sty, it might be a good idea to tweak the value of
% \extrarowheight as needed to properly center the text within the cells
\centering
\caption{Accuracy of K-NN vs Cascade of K-Means and K-NN}
\label{table_result}
%% Some packages, such as MDW tools, offer better commands for making tables
%% than the plain LaTeX2e tabular which is used here.
\begin{tabular}{|c|c|c|c|}
\hline
No. & Methods & Accuracy\\
\hline
1 & K-NN & 87\%\\
\hline
2 & Cascade of K-Means and K-NN & 88\%\\
\hline
\end{tabular}
\end{table}


% \includegraphics[width=3.5 in]{fig2b}
% \\
% \includegraphics[width=3.5 in]{fig2c}
% \caption{U Matrix (5x1996608) - One of the Output of FCM and Cluster Index Matrix (1x1996608)}
% \label{fig_sim}
% \end{figure}

%\subsubsection{Subsubsection Heading Here}
%Subsubsection text here.


% An example of a floating figure using the graphicx package.
% Note that \label must occur AFTER (or within) \caption.
% For figures, \caption should occur after the \includegraphics.
% Note that IEEEtran v1.7 and later has special internal code that
% is designed to preserve the operation of \label within \caption
% even when the captionsoff option is in effect. However, because
% of issues like this, it may be the safest practice to put all your
% \label just after \caption rather than within \caption{}.
%
% Reminder: the "draftcls" or "draftclsnofoot", not "draft", class
% option should be used if it is desired that the figures are to be
% displayed while in draft mode.
%
%\begin{figure}[!t]
%\centering
%\includegraphics[width=2.5in]{myfigure}
% where an .eps filename suffix will be assumed under latex, 
% and a .pdf suffix will be assumed for pdflatex; or what has been declared
% via \DeclareGraphicsExtensions.
%\caption{Simulation Results}
%\label{fig_sim}
%\end{figure}

% Note that IEEE typically puts floats only at the top, even when this
% results in a large percentage of a column being occupied by floats.


% An example of a double column floating figure using two subfigures.
% (The subfig.sty package must be loaded for this to work.)
% The subfigure \label commands are set within each subfloat command, the
% \label for the overall figure must come after \caption.
% \hfil must be used as a separator to get equal spacing.
% The subfigure.sty package works much the same way, except \subfigure is
% used instead of \subfloat.
%
%\begin{figure*}[!t]
%\centerline{\subfloat[Case I]\includegraphics[width=2.5in]{subfigcase1}%
%\label{fig_first_case}}
%\hfil
%\subfloat[Case II]{\includegraphics[width=2.5in]{subfigcase2}%
%\label{fig_second_case}}}
%\caption{Simulation results}
%\label{fig_sim}
%\end{figure*}
%
% Note that often IEEE papers with subfigures do not employ subfigure
% captions (using the optional argument to \subfloat), but instead will
% reference/describe all of them (a), (b), etc., within the main caption.


% An example of a floating table. Note that, for IEEE style tables, the 
% \caption command should come BEFORE the table. Table text will default to
% \footnotesize as IEEE normally uses this smaller font for tables.
% The \label must come after \caption as always.
%



% Note that IEEE does not put floats in the very first column - or typically
% anywhere on the first page for that matter. Also, in-text middle ("here")
% positioning is not used. Most IEEE journals use top floats exclusively.
% Note that, LaTeX2e, unlike IEEE journals, places footnotes above bottom
% floats. This can be corrected via the \fnbelowfloat command of the
% stfloats package.


\section{Conclusion}

Entropy and energy of GLCM are the features that were extracted for texture image classification in this work. Using splitting data, in which 50\% used for training and 50\% for testing, the system for classification was built. The system result shows taht accuracy of pure K-NN is 87\%. While the accuracy of cascade K-Means and K-NN is 88\%, meaning it increases the accuracy at 1\%. 

% Other than that conclusion, based on the experience of this work, the accuracy really depends on the feature selection and the similarity of image texture. If the textures are quite similar, the accuracy might be dropped. On the other hand, if the textures similarities between each classes are quite distinc, the accuracy may dropped. 

% if have a single appendix:
%\appendix[Proof of the Zonklar Equations]
% or
%\appendix  % for no appendix heading
% do not use \section anymore after \appendix, only \section*
% is possibly needed

% use appendices with more than one appendix
% then use \section to start each appendix
% you must declare a \section before using any
% \subsection or using \label (\appendices by itself
% starts a section numbered zero.)
%


%\appendices
%\section{MATLAB Code}
%\lstinputlisting{matlab_anfis/DCT_GLCM_ANFIS.m}
%\lstinputlisting{matlab_anfis/example2.m}
%\lstinputlisting{fcm_ari/clstidx.m}

% you can choose not to have a title for an appendix
% if you want by leaving the argument blank


% use section* for acknowledgement
\section*{Acknowledgment}

The authors would like to thank Dr. Widodo Budiharto of Bina Nusantara University, Jakarta, Indonesia for his guidance.
% The author would like to thank Dr. Bayram Cetiþli Suleyman of Demirel University, Computer Engineeering, Isparta, Turkey for his work in the ANFIS MATLAB program that was used and modified in this study.  


% Can use something like this to put references on a page
% by themselves when using endfloat and the captionsoff option.
\ifCLASSOPTIONcaptionsoff
  \newpage
\fi


% trigger a \newpage just before the given reference
% number - used to balance the columns on the last page
% adjust value as needed - may need to be readjusted if
% the document is modified later
%\IEEEtriggeratref{8}
% The "triggered" command can be changed if desired:
%\IEEEtriggercmd{\enlargethispage{-5in}}

% references section

% can use a bibliography generated by BibTeX as a .bbl file
% BibTeX documentation can be easily obtained at:
% http://www.ctan.org/tex-archive/biblio/bibtex/contrib/doc/
% The IEEEtran BibTeX style support page is at:
% http://www.michaelshell.org/tex/ieeetran/bibtex/
%\bibliographystyle{IEEEtran}
% argument is your BibTeX string definitions and bibliography database(s)
%\bibliography{IEEEabrv,../bib/paper}
%
% <OR> manually copy in the resultant .bbl file
% set second argument of \begin to the number of references
% (used to reserve space for the reference number labels box)
\bibliographystyle{IEEEtran}   % Probeer ook abbrv of alpha ipv plain
\bibliography{ANFIS}

% \begin{thebibliography}{1}

% \bibitem{Chandan}
% \newblock P. H. Chandankhede, 
% \newblock “Soft Computing Based Texture Classification with MATLAB Tool,” \textit{Int. J. Soft Compting Eng.}, no. 2, pp. 475–480, May 2012.

% \bibitem{Mathworks1}
% MathWorks-Doc.
% \newblock {"Train Adaptive Neuro-Fuzzy Inference Systems (GUI)."}
% \newblock {Internet: http://www.mathworks.com/help/
% fuzzy/train-adaptive-neuro-fuzzy-
% inference-systems-gui.html, [May 10, 2015]}

% \bibitem{Mathworks2}
% MathWorks-Doc.
% \newblock {"Neuro-Adaptive Learning and ANFIS."}
% \newblock {Internet: http://www.mathworks.com/help/fuzzy/neuro-adaptive-learning-and-anfis.html, [May 23, 2015]}

% \bibitem{Dataset}
% Mario Fritz, Eric Hayman, Barbara Caputo and Jan-Olof Eklundh. 
% \newblock {"THE KTH-TIPS database."}
% \newblock {Internet: http://www.mathworks.com/help/fuzzy/train-adaptive-neuro-fuzzy-inference-systems-gui.html, May 7, 2004, [May 10, 2015]}

% \bibitem{StackOverFlow2014}
% StackOverFlow Forum.
% \newblock {"How to Calculate the Energy and Correlation of An Image."}
% \newblock [Online]. Avalable: http://stackoverflow.com/questions/22953557/how-to-calculate-the-energy-and-correlation-of-an-image. April 14, 2014, [May 20, 2015]

% \bibitem{Cruz}
% A. O. Cruz, “ANFIS : Adaptive Neuro-Fuzzy Inference Systems ANFIS Architecture Hybrid Learning Algorithm ANFIS as a Universal Approximatior Simulation Examples.” [Online]. Available: http://equipe.nce.ufrj.br/adriano/fuzzy/transparencias/anfis/anfis.pdf. [Accessed: 20-May-2015].



% \end{thebibliography}

% biography section
% 
% If you have an EPS/PDF photo (graphicx package needed) extra braces are
% needed around the contents of the optional argument to biography to prevent
% the LaTeX parser from getting confused when it sees the complicated
% \includegraphics command within an optional argument. (You could create
% your own custom macro containing the \includegraphics command to make things
% simpler here.)
%\begin{biography}[{\includegraphics[width=1in,height=1.25in,clip,keepaspectratio]{mshell}}]{Michael Shell}
% or if you just want to reserve a space for a photo:

\begin{IEEEbiographynophoto}{B. Ari Kuncoro}
recieved B. Eng. (Sarjana Teknik) in Electrical Engineering from Bandung Institute of Technology in 2008. From 2009 to 2015, he was an intelligent networks engineer in a global telecommunication company, handled charging system of pre-paid mobile subscribers and was a software quality engineer in an Indonesian microfinance company. Currently he is a master student of Bina Nusantara University Graduate Program, Jakarta in Information Technology. His research interest includes Artificial Intelligence, Image Processing, and Data Mining. 
\end{IEEEbiographynophoto}



% if you will not have a photo at all:
%\begin{IEEEbiographynophoto}{John Doe}
%Biography text here.
%\end{IEEEbiographynophoto}

% insert where needed to balance the two columns on the last page with
% biographies
%\newpage

%\begin{IEEEbiographynophoto}{Jane Doe}
%Biography text here.
%\end{IEEEbiographynophoto}

% You can push biographies down or up by placing
% a \vfill before or after them. The appropriate
% use of \vfill depends on what kind of text is
% on the last page and whether or not the columns
% are being equalized.

%\vfill

% Can be used to pull up biographies so that the bottom of the last one
% is flush with the other column.
%\enlargethispage{-5in}



% that's all folks
\end{document}


